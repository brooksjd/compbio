\documentclass[12pt,twocolumn]{article}
\usepackage{fullpage}
\usepackage{amsmath}
\usepackage{graphicx}
\usepackage{enumerate}
\usepackage{float}
\usepackage{listings}
\usepackage{longtable}
\usepackage[table]{xcolor}
\usepackage{tabularx}
\usepackage{parskip}
\usepackage[round]{natbib}
\restylefloat{figure}
\title{Gamifying the Transcriptome}
\author{Chidube Ezeozue and Joel Brooks}

\begin{document}
\bibliographystyle{apalike}
\renewcommand\refname{Bibliography}
\maketitle

\begin{abstract}

\end{abstract}

\section*{Introduction}

\section*{Methods}

\subsection*{Data acquisition and preparation}

\subsection*{Puzzle selection and construction}

\subsection*{Game implementation}
The game interface is built using the LimeJS HTML5 game framework. This framework provided methods for creating shapes,
animations, and handling user interaction events, allowing us to focus more on developing the core functionality of the game.
Additionally, the fact that the game is implemented in HTML5 and javascript means that it can be played natively in most 
popular browsers, as well as both desktop and mobile hardware.

We also made use of Kenneth Kelly's list of 22 colors of maximum contrast \citep{green2010colour}. This allowed us to assign a unique color each exon
in a particular puzzle that was easily distinguishable from the colors of all the other exons. This allowed us to present genes to 
users in a manner that was both visually appealing and functional.

\subsection*{Solution scoring}

\section*{Results}

\subsection*{Gameplay statistics}

\subsection*{Player solutions}

\section*{Discussion}

\bibliography{frbib}

\end{document}
